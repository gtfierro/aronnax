\section{Implementation} \label{section:implementation}

Here, we discuss the implementation of the temporal query language over a
temporal entity-attribute-query MySQL table. The project is under active
development and is available
online\footnote{\url{https://github.com/gtfierro/aronnax/tree/master/sql}}.
While the implementation is not yet complete, it illustrates the nontriviality
of incorporating time with an EAV table, and some of the limits of SQL.

Our implementation is based around a compiler that translates our temporal
query language into a SQL query. The compiler is written in the Go~\cite{go} version
of Yacc~\cite{johnson1975yacc}\cite{goyacc}.

The current implementation supports automatic server-generated timestamps (provided by MySQL's \texttt{TIMESTAMP}
data type) as well as voluntary client-generated timestamps. The temporal entity-attribute-value MySQL schema is as
follows:
\begin{sqlcode}
CREATE TABLE data
(
    uuid CHAR(37) NOT NULL,
    dkey VARCHAR(128) NOT NULL,
    dval VARCHAR(128) NULL,
    timestamp TIMESTAMP NOT NULL
);
\end{sqlcode}

We now build up the set of subqueries required to implement the temporal query language.

\subsection{Basic Relational Predicates}

We begin by simplifying the problem to executing a non-temporal relational predicate on only current
metadata. This is more complex than simply emulating relational logic on an EAV table, because each
predicate must identify the most recent version of each \texttt{<uuid, dkey, dval>} tuple from the full
stored history. The query for this is straightforward:

\begin{sqlcode}
select distinct uuid, dkey, max(timestamp)
from data
group by dkey order by timestamp desc;
\end{sqlcode}

We now want to filter the set of tuples by those that match the provided relational predicate. For this
example, we'll use the equality predicate \texttt{Location/Room = '410'}. The following query, which is
the compiled form of \texttt{select * where Location/Room = '410'} returns
all tuples for each stream (\texttt{uuid}) in a tuple that matches the predicate:

\begin{sqlcode}
select data.uuid, data.dkey, data.dval
from data
inner join
(
    select distinct uuid, dkey, max(timestamp) as maxtime
    from data
    group by dkey, uuid order by timestamp desc
) filtered
on
    data.uuid = filtered.uuid
    and data.dkey = filtered.dkey
    and data.timestamp = filtered.maxtime
where data.dval is not null
and (data.dkey = "Location/Room" and data.dval = "410");
\end{sqlcode}

The relational predicates must be applied \emph{after} the temporal filtering,
otherwise the generated query will return streams that matched the predicate at
any point in the past.

Next, we build up an implementation of \syntax{`AND'} and \syntax{`OR'}. \syntax{`AND'}
is an intersection (\texttt{INNER JOIN}) between two sets; thus we can implement \syntax{`AND'}
as an inner join on \texttt{uuid} between two subqueries like the above:

\begin{sqlcode}
SELECT A.uuid FROM
(<subquery 1>) as A
inner join
(<subquery 2>) as B
on
A.uuid = B.uuid;
\end{sqlcode}

\syntax{`OR'} is a union (\texttt{UNION}) between two sets, leading to a straightward implementation:

\begin{sqlcode}
SELECT uuid FROM
(<subquery 1>) as A
union
(<subquery 2>);
\end{sqlcode}

These constructions of \syntax{`AND'} and \syntax{`OR'} are easily combinable:

\begin{sqlcode}

SELECT A.uuid FROM
(<subquery 1>) as A
inner join
(<subquery 2>) as B
on A.uuid = B.uuid
union
(
  SELECT C.uuid FROM
  (<subquery 3>) as C
  inner join
  (<subquery 4>) as D
  on C.uuid = D.uuid
);
\end{sqlcode}

\subsection{Temporal Predicates}

One of the advantages of the generated queries for the basic relational model above is that they extend naturally to incorporate
the temporal predicates. For most of these, the implementation can be done entirely in the innermost \syntax{<select>}
clause for each subquery.

\texttt{AT} is implemented by restricting the ``current'' query to all facts inserted before the given time:

\begin{sqlcode}
select distinct uuid, dkey, max(timestamp) as maxtime
from data where timestamp <= 1234567890
group by dkey, uuid order by timestamp desc
\end{sqlcode}

For \texttt{HAPPENS BEFORE} and \texttt{HAPPENS AFTER} we remove the \texttt{max} aggregator to return all facts that meet the predicate
before or after the given timestamp:

\begin{sqlcode}
select distinct uuid, dkey, timestamp as maxtime
from data
where timestamp < 1234567890 -- or > for AFTER
order by timestamp desc
\end{sqlcode}

\texttt{IN} is a natural combination of these two, but the implementation of
\texttt{FOR} is slightly more copmlicated because we must doubly assert that
not only is the fact is true between the two timestamps but also that it does not
become false. To implement \texttt{FOR}, we must assert that the predicate is
true at the lower bound of our time range and that though there may be changes
involving the \texttt{<uuid, key, value>} pair concerned by the predicate, none
of those changes invalidate the predicate. This is necessary for predicates
involving \syntax{`!='}, \syntax{`LIKE'} or \syntax{`HAS'}.

\subsection{Reflection}

Although this proof-of-concept implementation is not complete, the length and
complexity (see the Appendix) of the generated SQL queries suggests that there is an opportunity
for a more appropriate solution. The general expressiveness of the full
relational model becomes much more complex when time is added as a dimension.
Temporal query languages such as TQuel~\cite{snodgrass1987temporal},
TSQL2~\cite{snodgrass2012tsql2} and Tquery~\cite{kahn1991tquery} define clear
semantics for time-based joins across relations, valid and transactional times,
which do not apply to our simpler metadata model.
Additiionally, these do not permit decoupled temporal predicates for both
\syntax{<select>} and \syntax{<where>} clauses, which our model requires for
querying across two separate databases.


%\begin{sqlcode}
%-- WHERE <pred> FOR (A, B)
%WHERE <pred> at A and not <~pred> happens in (A, B)
%\end{sqlcode}
