\section{Metadata Model}

In this section, we describe a natural model for streams of physical timeseries
data and its associated metadata. This model is our framework for exploring
temporal metadata. We then discuss the set of possible database representations
of the metadata, ultimately choosing an Entity-Attribute-Value
(EAV)~\cite{chen1976entity} model. Then, we briefly review a query language for
this model, based on sMAP~\cite{dawson2010smap}, designed to retrieve both
timeseries data and metadata.


\subsection{Timeseries and Metadata}

A \emph{stream} is a virtual representation of a specific sequence of data; for
a physical timeseries, this might be a sensor or actuator channel. Each stream
is uniquely identified by a UUID, and consists of a single progression of
\texttt{<timestamp, value>} pairs and an unordered set of key-value pairs
(metadata).

Telemetry databases such as Prometheus~\cite{prometheus} and
InfluxData~\cite{influxdata} will often use a stream's description as its
unique identifier and further qualify it with a small series of tags; for
example a stream named \texttt{total\_http\_requests} might have a tag of
\texttt{hostname=host1.example.com}. With this construction, the database makes
the assumption that changing any aspect or context of the stream warrants the
creation of an entirely new timeseries stream with a new identifier.

The decision at hand is whether streams should be identified by \emph{what they
measure} or by \emph{who produces them}. Our model takes the latter approach:
using a stream UUID as the unit of association enables a more flexible approach
to how data is described and how it is collected.

\subsection{Data Representations}

Here, we discuss how this model is represented in a database, which influences
how a temporal dimension can be added. We simplify the representation of our
model by separating the storage of timeseries data from metadata. Timeseries
data is placed in a store optimized specifically for timeseries data; stream
UUIDs are used to associate timeseries streams with their metadata. For
storing metadata, we have three options -- denormalized (NoSQL), vertical
and horizontal relational (normalized).

For our model, we choose a vertical database representation, also known
as an Entity-Attribute-Value (EAV) model~\cite{chen1976entity}, containing
three columns: the stream UUID, a metadata key, and a metadata value. This
structure is appropriate because there is no strict schema for what metadata
keys a stream can have, which suggests a sparse representation. While metadata
models more complex than an unordered set of key-value pairs exist (e.g. XML),
most can be reduced to a key-value representation.

Given this structure, a denormalized representation also seems appropriate,
but as we will see in Section~\ref{section:temporal}, when the dimension of time
is included, insertion on an EAV table becomes append-only. EAV tables can also
naturally extend to include a fourth column for a timestamp; such a transformation
is non-obvious in a denormalized database.

%we append only, but even though we can add timestmaps in the past, we can
%have a sort of concurrent garbage collection that attempts to reorder the table.
%We can edit the table in place, because if there are 2 copies of the record,
%all queries will still complete correctly.

\subsection{Non-Temporal Query Language}

\begin{itemize}
\item First establish what our normal timeseries and metadata looks like
  \begin{itemize}
  \item document UUID, bag of key-value pairs
  \item Overview of how data gets deposited: what is current system's opinion on timestamps?
  \item What does the current ``query language'' for these look like:
      \begin{itemize}
      \item selecting timeseries data based on metadata
      \item selecting metadata based on metadata
      \end{itemize}
  \item Data representation: EAV table
  \end{itemize}
\end{itemize}
