\section{Motivating Applications}

As the prices and sizes of sensors drop, there has been a dramatic increase in
the production of timeseries data. Streams of timeseries data, represented as
sets of \texttt{<time, value>} pairs, can give applications insights into the
instantaneous or historical behavior of the physical world.  However, these
streams of data are only as useful as their contextual information.  This
contextual information, or \emph{metadata}, contains all data that describes a
stream, often including at least one of the following:

\begin{itemize}
\item a unique identifier (UUID)
\item timezone for the stream's timestamps
\item engineering units for the stream's values
\item the unit of time or sampling rate
\item location (building, floor, room, orientation, etc)
\item calibration constants
\item software versions
\end{itemize}

There has already been substantial research on the application of raw
timeseries data in the social sciences~\cite{mccleary1980applied},
econometrics~\cite{lutkepohl2004applied}, electric vehicle
charging~\cite{sortomme2011optimal}, electric
grids~\cite{carreras2004evidence}, building occupancy~\cite{richardson2008high}
and fault detection~\cite{fontugne2013strip}. Common to all of these
applications of timeseries data is the tacet assumption that a data stream,
once identified or discovered, will retain consistent metadata. For some circumstances,
this assumption is valid, but in the domain of physical timeseries data (often
produced by sensors), this is often not the case.

Sensors will often change

Current applications telemetry platforms and time series databases do not
account for the inevitable evolutions in metadata for the streams they operate
on. Here, we describe a family of applications that require the ability to
perform queries at instantaneous moments in the past or over ranges over time.

\todo{this goes under description of our structures}

Telemetry databases such as Prometheus~\cite{prometheus} and
InfluxData~\cite{influxdata} will often use a stream's description as its
unique identifier and further qualify it with a small series of tags; for
example a stream named \texttt{total\_http\_requests} might have a tag of
\texttt{hostname=host1.example.com}. With this construction, the database makes
the assumption that changing any aspect/context of the stream warrants the
creation of an entirely new timeseries stream with a new identifier. 

This is fine for telemetry data, but when we consider the more general case of
physical timeseries data, we require a different definition of \emph{stream
provenance}.

We do not wish to propose a solution to Theseus' Paradox: for our purposes, a UUID is the definition of
a data source
