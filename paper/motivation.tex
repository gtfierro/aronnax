\section{Motivating Applications}

As the prices and sizes of sensors drop, there has been a dramatic increase in
the production of timeseries data. Streams of timeseries data, represented as
sets of \texttt{<time, value>} pairs, can give applications insights into the
instantaneous or historical behavior of the physical world.  However, these
streams of data are only as useful as their contextual information.  This
contextual information, or \emph{metadata}, contains all data that describes a
stream, often including at least one of the following:

\begin{itemize}
\item a unique identifier (UUID)
\item timezone for the stream's timestamps
\item engineering units for the stream's values
\item the unit of time or sampling rate
\item location (building, floor, room, orientation, etc)
\item calibration constants
\item software versions
\end{itemize}

There has already been substantial research on the application of physical
timeseries data for electric vehicle charging~\cite{sortomme2011optimal},
electric grids~\cite{carreras2004evidence}, building
occupancy~\cite{richardson2008high} and fault
detection~\cite{fontugne2013strip}. Common to all of these applications of
timeseries data is the tacet assumption that a data stream, once identified or
discovered, will retain consistent metadata. For some circumstances, this
assumption is valid, but in the domain of physical timeseries data (often
produced by sensors), this is often not the case. Metadata can change due to
changes in location or orientation, repairing configuration error, or changes
in the deployment site or environment.
% need to be more clear on what those circumstances are, exactly. this is linked
% to the identity of timeseries streams

Current applications telemetry platforms and time series databases do not
account for the inevitable evolutions in metadata for the streams they operate
on. Here, we describe a family of applications that require the ability to
perform queries at instantaneous moments in the past or over ranges over time.
These applications fall into three categories, each of which places its
own requirements on the underlying metadata storage and retrieval system:
offline analysis of timeseries data, analysis of metadata, and real-time
metadata-based streaming.

\if 0
For each of these, we want to list what the possible applications are,
the kinds of queries they will require, the functionality that they require,
and why current solutions do not make this easy. also maybe have a figure
on what these queries would look like, and maybe talk about how they aren't possible
without temporal data.
\fi

\subsection{Timeseries Analysis}

Without a temporal dimension, metadata describing a timeseries can only
capture a single static context, which may not be consistent over the
whole timeseries. Adding duration to metadata means there is no invalidation
of prior data when metadata does change, enabling downstream consumers
of the timeseries data to make use more fine-grained descriptions.

There are two sample timeseries analysis applications we discuss here: 1) an
application that computes the average monthly energy usage for a building that
must account for a calibration correction partway through data collection, and 2)
a fault detection application that wants to ``tag'' faulty ranges of timeseries data.


\if 0
Figure here: show some timeseries data (maybe a building feed w/ a configuration
constant change?). App needs to know to correct the data.

another figure: a temperature sensor moving from room to room?
figure: a plugstrip where what is plugged into it changes
\fi

\if 0
have some figure here,
- list of applications:
    - sensor move (floor, room, orientation, timezone)
    - software hanges
    - reporting rates change
    - calibration constants change
    - tagging of transient events: [T1, T2] was a voltage sag

Two benefits:
1. you know that the tags are correct and not just changed
2. you can tag events that you discover!
\fi

\subsection{Metadata Analysis}

%- treat metadata as data
%    - all rooms a sensor was in over past month
%    - all sensors that were in room XYZ at the time this event happened
%    - the fault detection app from above, but it wants to look at the history of detected faults
\subsection{Real-Time}

%Motivating Applications
%- realtime change monitoring
%    - control process takes average of all sensors in rooms 1,2,3
