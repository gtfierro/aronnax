\section{Introduction}

\if 0
This is essentially a summary of the paper, so do a good job here.
\fi

Trends in Big Data, the dropping prices and sizes of sensors and the increasing
ease of monitoring and recording every possible metric mean that there is more
timeseries data now than ever before.  Streams of timeseries data, represented
as sets of \texttt{<time, value>} pairs, can give applications insights into
the instantaneous or historical behavior of the physical world.  There is
substantial research on the application of timeseries data across many fields
including the social sciences~\cite{mccleary1980applied},
econometrics~\cite{lutkepohl2004applied} and extending into the physical world
for electric vehicle charging~\cite{sortomme2011optimal}, electric
grids~\cite{carreras2004evidence}, building occupancy~\cite{richardson2008high}
and fault detection~\cite{fontugne2013strip}.

Though it is simple to store large volumes
of timeseries data, the data is only as useful as its contextual information.
This contextual information, or \emph{metadata}, contains all data that describes
a stream of timeseries data, often including at least one of the following:

\begin{itemize}
\item a unique identifier (UUID)
\item timezone for the stream's timestamps
\item engineering units for the stream's values
\item the unit of time or sampling rate
\item location (building, floor, room, orientation, etc)
\item calibration constants
\item software versions
\end{itemize}


Most timeseries database systems that deal with metadata  make the tacet
assumption that a data stream, once identified or discovered, will retain
consistent metadata. This simplified model is insufficient for the many
timeseries (such as those produced by sensors) that do evolve over time.
Metadata can change due to changes in location or orientation, repairing
configuration error, changes in the deployment site or environment or even
changes in server and software configurations. More generally, pervasive
computing requires temporal contextual information, which can become stale over
time~\cite{henricksen2002modeling}.

In this paper, we define a combined model for timeseries data and static
metadata and discuss how to extend it temporally to describe and store varying
metadata. This model is contrasted with more general relational temporal models
from the literature. We then design a temporal query language for the proposed
combined timeseries data and temporal metadata, and discuss a proof-of-concept
implementation written in Go using Yacc over an entity-attribute-value-time MySQL
table.
