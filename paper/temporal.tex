\section{Temporal Metadata Model} \label{section:temporal}

In this section, we extend our metadata model with a temporal dimension that
enables the applications in Section~\ref{section:motivation}. To do this effectively, we first
discuss our notions of \emph{time} and how to mitigate discrepancies between client and server.
We then examine established temporal data models such as the Bitemporal Conceptual Data Model (BCDM).
Because most research into temporal data models has been conerned with relational or spatial models,
these models are more complicated than needed for a simple EAV table.

\begin{itemize}
\item then start talking about temporal query languages?
\item temporal query language tquel~\cite{snodgrass1987temporal}
\item tsql2~\cite{snodgrass2012tsql2}
\item tquery~\cite{kahn1991tquery}
\item differentiate from continuous query languages like CQL~\cite{arasu2006cql}
\end{itemize}

\subsection{What is Time?}

There are two flavors of time that temporal database must deal with:
\emph{valid time} and \emph{transaction time}~\cite{jensen1999temporal}. 

\textbf{Valid time} is the extent for which a given \emph{fact} is true. For
our model, a fact is the association of a key-value pair with a stream,
changing the value associated with a stream's key, or removing a key-value pair
from a stream's metadata. Valid time is expressed in terms of real-world time
and is how metadata facts are aligned with timeseries data, which is stored
using real-world time. Because valid time is the coordinating aspect across the
two databases (timeseries and metadata), it is also the point of interaction
for consumers of data.

A fact's valid time can be qualified either by a single time (``start'') or by
a duration (``start'' and ``end''). The use of a start-time lends itself to
diff-based storage, where facts for each \texttt{<uuid, key>} pair are
considered valid until a new fact is written for the same \texttt{<uuid, key>}
pair. This also removes the need to invalidate past facts, which is a drawback
of the bitemporal model (BCDM).

\textbf{Transaction time} is a logical timestamp that represents when a fact
takes place in relation to other facts. In a temporal database where all
entries are made at the current time, transaction time can be conflated with
valid time, but if the database allows retroactive inserts, then these two
times must be separate. In our demo implementation, transaction time is made
implicit through the use of transactions on the underlying database. Unlike
valid times, which maybe added in the past, transaction times are strictly
monotonic.

\if 0
have a figure here? start-time vs duration time?
\fi

\subsection{Temporal Data Models}

\begin{itemize}
\item BCDM bitemporal conceptual data model: only captures when facts are valid in reality and when stored in database~\cite{jensen1996semantics}\cite{jensen1994unifying}
\item incremental relational temporal model: \cite{jensen1991incremental}
\item relational algebra accounts for transaction time, evolution of contents as well as schema~\cite{mckenzie1990schema}
\end{itemize}

%[39, 52, 53, 74]

%\begin{itemize}
%\item Discuss current tmporal data models
%    \begin{itemize}
%    \item TODO: make this list
%    \item OURS: how to augment with time: add column to EAV
%    \end{itemize}
%\item discuss sql vs nosql for demo implementation?
%    \begin{itemize}
%    \item SQL more suited for diffs?
%    \item NoSQL more suited for replication?
%    \end{itemize}
%\end{itemize}
%
Some related work:
\begin{itemize}
\item $<tt, vt>$ (timestamp, validtime) for temporal data~\cite{jensen1999temporal},
also stratum vs integrated approaches for implementing temporal databases
\item extending relational algebra for temporal data?~\cite{lorentzos1988extending}
\item list of temporal terms~\cite{dyreson1994consensus}
\end{itemize}
